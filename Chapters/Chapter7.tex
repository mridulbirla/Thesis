\doublespacing
\chap{Conclusion}
\label{chap:Conclusion}

The main aim of this work to carefully study, understand, document the generative adversarial network from ground up. We start with understanding the  convolutional neural network as it is one the major component of this work. In \autoref{chap:CNN} we study 101 of CNNs. Later in \autoref{chap:RelatedWork}, we look at the previous generation models. Since the GAN are currently a hot topic in field of depp learning and computer vision. This can be seen from many framework being proposed in recent years and we look at some of them in \autoref{chap:RelatedWork}. By looking at some of the work we started with conditional generation of the image as this is one of the key sub-problem for many large problem such as generating images from caption, generating complex captcha. In \autoref{chap:EGAN} we take propose the  tweaks and tricks that many authors failed to mention in their research. During experimentation phase we faced lot of challenges in producing any kind of images but with techniques proposed in \autoref{chap:EnR} we proposed some approaches we were able to generate images in very less epochs.
\par
In the end, there is still a long way to have a stabilized and reliable method for generating images using GAN. There is also need for a better way to quantify the  performance of a GAN framework.