\doublespacing
\chap{Conclusion}
\label{chap:Conclusion}

The main aim of this work is to carefully study, understand, document the generative adversarial networks from the ground up. We start with understanding the  convolutional neural network as it is one the major components of this work. In \autoref{chap:CNN} we study CNNs. Before introducing the basic concepts of GAN we look at the previous generation models, since GANs are currently a hot topic in the field of deep learning and computer vision. This can be seen from many framework being proposed in recent years and we look at some of them in \autoref{chap:RelatedWork}. By studying some of the work we started with conditional generation of images as this is one of the key sub-problems for many large problem such as generating images from caption, generating complex captcha, super-resolution of images. In \autoref{chap:EGAN} we take propose the  tweaks and tricks that many authors failed to mention in their research. During experimentation phase we faced lot of challenges in producing any kind of images but with techniques proposed in \autoref{chap:EnR} we proposed some approaches we were able to generate images in very less epochs.
\par
In the end, there is still a long way to create a stabilized and reliable method for generating images using GAN. GAN also have shown that the CNNs for the object recognition now have to deal with fake images also. There is also need for a better way to quantify the  performance of a GAN framework.